\section{Conclusions}
\label{sec:conc}

This paper has introduced the new WISDEM module, \textit{FloatingSE},
for hydrostatic-based sizing and conceptual design of floating offshore
wind turbine substructures.  To showcase the capabilities of this
module, and the larger WISDEM framework, a two-step analysis was carried
out.  First, three different substructures, a spar, semisubmersible, and
TLP, were designed for the DTU \unit[10]{MW} reference turbine using the
same set of descriptive configuration variables and analysis tools.  This
demonstrates the ability of \textit{FloatingSE} to parametrically
describe entirely different platform architectures that achieve
stability through different underlying mechanisms.  Second, a design
sensitivity study was conducted where mass in the nacelle was
parametrically removed, to simulate the addition of a novel drivetrain
or generator technology, and the design re-optimized.  The derived
sensitivities were used to ascertain the break-even cost rate of the new
technology that reduces the drivetrain mass.

The optimization-based analyses yielded a few interesting, and some
unexpected results.  First, cost reductions in floating offshore wind
energy can be achieved by a consideration of the entire engineering,
manufacturing, and operation requirements concurrently.  Second, mass
can be a poor surrogate for cost in engineering design as due to
differences in cost rates and complexity in cost models.  This
underscores the need for a cost-focused multidisciplinary systems
framework for floating offshore wind.  Finally, the break-even point for
drivetrain technologies that offer mass savings for a cost premium are
approximately \$1,000 for the spar, \$450 for the semisubmersible, and
\$100 for the TLP.

The design sensitivity study was just one of the questions that could be
posed of \textit{FloatingSE} and WISDEM to help answer one of the many
outstanding questions regarding floating offshore wind technology.  Some
of the other questions that could be explored in greater depth include,
\begin{itemize}
\item What are the cost-benefit tradeoffs of a floating substructure
  designed to operate in many different regions versus one that is more
  customized to particular metocean environment?
\item What is the impact on floating systems (e.g., weight, cost,
  scale-ability) due to a new technology?  Is that new
  technology worth the investment?
\item What technologies should governments and industry invest in to
  achieve the greatest cost reduction in floating offshore wind energy?
\item Where can alternative materials, such as composites and concrete,
  be used on the turbine to reduce cost, taking into account regional
  material and labor cost differences?
\end{itemize}

The results presented here were prefaced with many caveats due to the
many simplifying assumptions and low-fidelity analyses within
\textit{FloatingSE}.  Hence, there is significant plans for future
improvements.  Future plans include accounting for hydrodynamics and
additional load cases and a more detailed cost model of the balance of
station and operational costs of an offshore floating wind plant in
WISDEM.  As these improvements are made, the results determined in this
paper may change, perhaps significantly.  Nevertheless, the improvements
on the whole will enable the framework to address a richer set of open
analysis questions with greater certainty.

\section*{Acknowledgments}
The original version of this model was developed by Senu Sirnivas, who
deserves credit for laying out the methodology described here.  The
authors would also like to thank Patrick Gilman, Alana Duerr, Gary
Norton, Daniel Beals, Eric Lantz, and Jason Jonkman for their valuable
input in reviewing results and drafts of this paper.

NREL’s contributions to this work were supported by the US Department of
Energy under Contract Number DE-AC36-08GO28308 with the National
Renewable Energy Laboratory.

