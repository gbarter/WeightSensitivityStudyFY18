\chapter{Analysis}
\label{sec:opt}

\section{Procedure}
Single scenario execution of \textit{FloatingSE} and/or WISDEM is
sufficient to explore some simple one-off or comparison analyses between
a few runs.  However, executing the model within an optimization
framework can give yield richer and more insightful analyses.  There are
two layers of optimization-based analyses presented in this work:

\begin{enumerate}
\item Use \textit{FloatingSE} and WISDEM to design a spar,
  semisubmersible, and TLP floating substructure for the same turbine
  configuration;
\item Conduct a sensitivity analysis around the optimized designs for
  weight reduction at the top of the tower by parametrically changing
  the mass of the nacelle and re-optimizing the substructure.
\end{enumerate}

\subsection{Substructure Design}
The first step in the analysis procedure is the conceptual design of a
spar, semisubmersible, and TLP floating substructure for the same
turbine configuration.  Many studies in the literature focus on the
optimization of a single substructure design.  By showcasing the design
of all three classical substructure archetype, we demonstrate the ability
of the tool to capture a broad tradespace.

Through its various modules, WISDEM is fully capable of optimizing the
entire turbine design, from the rotor through the drivetrain and down
through the tower and substructure.  However, in an effort to focus on
the capabilities of \textit{FloatingSE}, the turbine was held fixed to
the DTU \unit[10]{MW} reference definition throughout the substructure
design and sensitivity analysis.  The only exception being the
deliberate modification of the nacelle mass for the sensitivity study.
This isolated the design variables and constraints to be those
associated with the substructure exclusively.

\subsection{Sensitivity Studies}
Following the successful conceptual design of the floating
substructures, a sensitivity study was executed relating the mass at the top of the
tower to the mass and cost of the substructure.  This was accomplished
by parametrically changing the nacelle mass and re-optimizing the
substructure designs.  Specifically, the nacelle mass was changed from
its nominal value to the perturbations of, $[+10\%, -10\%, -25\%, -33\%,
  -50\%]$.  For the sensitivity study, the re-optimization was done locally,
using the previously optimized designs as a starting point, and not
searching over the global tradespace.  Then, by
comparing the baseline design to the re-optimized designs under the
parameterized mass changes, the cost value of mass savings can be
determined.  Meaning, given the quantified mass and cost removed from
the substructure relative to the baseline design, the cost value of the
mass savings in the nacelle can be quantified.  In this way, a systems
framework can determine with the cost premium for the nacelle mass
reduction can be recovered through savings as the change is propagated
throughout the rest of the system.

The parametric modification of the nacelle mass is not merely an
academic exercise to tease out design sensitivities.  There are many
examples of a new technology or innovation that offers a component mass
reduction at a cost premium.  By parametrically changing the nacelle
mass, the sensitivity study is capturing the implications of moving from a
multi-stage geared drivetrain to a single-stage or direct-drive
alternative, or from a permanent magnet generator to a superconducting
generator.  In both of these examples, the overall mass of the nacelle
would likely decrease, but the new technology would cost more to
implement over the legacy system.  Conducting an honest cost-benefit
tradeoff of this option requires the potential design changes be
propagated through the rest of the system.  WISDEM is an appropriate
framework for posing and answering this type of question.


\subsection{Reference Turbine Selection}
For all elements of the optimization studies, attention is given to the
substructure and the performance of \textit{FloatingSE}.  A reference
turbine design was needed as a starting point by which the conceptual
design and sensitivity studies could build upon.  The Danish Technical
University (DTU) \unit[10]{MW} reference design was chosen as the static
turbine to be used throughout this analysis.  This selection was made
because \unit[10]{MW} is representative of the most current offshore
turbine designs being installed (with fixed-bottom supports) by
developers.  Additionally, other research efforts, such as the LIFES50+
program, have already designed floating substructures for this turbine.
Using the same reference turbine allows for a direct comparison of the
designs produced here with the other designs documented in the
literature.  For a summary of the DTU \unit[10]{MW} reference turbine,
see \citet{dtu10mw}.


\subsection{Metocean Conditions}
In addition to choosing a reference turbine, it was also necessary to
choose the meteorological and oceanographic (metocean) conditions of the
environment.  As mentioned in Section \ref{sec:theory},
\textit{FloatingSE} only uses a single DLC, the maximum rotor thrust and
metocean loading, for concept evaluation.  The metocean condition
selected for this DLC is the same as that used in the IEA Wind Task 30
OC3 effort \citep{OC3}, and summarized in Table \ref{tbl:metocean}.

\begin{table}[htbp] \begin{center}
    \caption{Metocean conditions used as the design point in the design
      and sensitivity studies (adopted from \citet{OC3}).}
    \label{tbl:metocean}
          {\small
            \begin{tabular}{ l l } \hline
              \textbf{Parameter} & \textbf{Value} \\ \hline \hline
              Wind reference speed & \unit[11]{$m/s$} \\
              Wind reference height & \unit[119]{$m$} \\
              Water depth & \unit[320]{$m$} \\
              Significant wave height & \unit[10.8]{$m$} \\
              Significant wave period & \unit[9.8]{$s$} \\ \hline
            \end{tabular}
          }
\end{center} \end{table}


\section{Methodology}
In this project, optimization studies start with the formulation of
a constrained, nonlinear single-objective optimization problem with
mixed-integer design variables,
\begin{equation}
\begin{array}{ll}
  \min & f\left(\mbf{x}\right)\\
  \text{subject to} & \mbf{g}\left(\mbf{x}\right) \leq 0,\\
  \text{and}& \mbf{x} \in \mbf{X} \\
  \end{array}
\end{equation}
where,
\begin{itemize}
\item $\mbf{x}$ is a vector of $n$ \textit{design variables}, the variables that are adjusted in order to
  find the optimal solution (see Table \ref{tbl:designvar});
\item $f(\mbf{x})$ is the nonlinear \textit{objective function}, the
  metric to be minimized by the optimization algorithm;
\item $\mbf{g} (\mbf{x})$ is the vector of \textit{inequality constraints}, the
  set of conditions that the solution must satisfy (see Table
  \ref{tbl:constraints}).  There are no equality constraints;
\item $\mbf{X}$ is the design variable \textit{bounds}, the bracket of
  allowable design variable values.
\end{itemize}

Note that this problem statement imposes no requirements on the types of
variables in $\mbf{x}$.  A mixed-integer solution is desired, where some
design variables are continuous ($x \in \mbb{R}$) and others are
discrete variables that can only take integer values ($x \in
\mbb{Z}$).  An example of an integer design variable in this
application is the number of offset columns or the number of mooring
line connections.


\subsection{Objectives}
As described above, two objective functions are used in this analysis:
\begin{itemize}
\item \textbf{Substructure system mass:} The mass of components from the
  freeboard point and below \textit{except} mooring and anchor system masses;
\item \textbf{Substructure system cost:} The capital cost of components from the
  freeboard point and below \textit{including} mooring and anchor system costs;
\end{itemize}


\subsection{Algorithms}
Two derivative-free optimization algorithms were applied sequentially to
the substructure design problem:
\begin{enumerate}
\item \textbf{Global design space search and optimization:} Native
  implementation of the Non Sorting Genetic Algorithm (NSGA)-II genetic
  algorithm \citep{nsga2} with some modifications for constraint and
  integer design variable handling;
\item \textbf{Local neighborhood design space optimization:} Native
  implementation of the Nelder-Mead simplex algorithm \citep{neldermead}  with some
  modifications for constraint handling.
\end{enumerate}

The global design space search was performed by a modified
implementation of the popular NSGA-II \citep{nsga2} algorithm.  This algorithm is
a non-dominating sorting genetic algorithm that solves non-convex and
non-smooth single and multiobjective optimization problems. The
algorithm attempts to perform global optimization, while enforcing
constraints, using a tournament selection-based strategy where the best
solutions are placed into a mating pool.  The in-house modifications of the
NSGA-II algorithm include parallelization via multi-threading for faster
execution, the use of penalties for constraint handling (described
below), and the handling of integer-based design variables for a fully
mixed-integer capable solution.  Specifically, a method for
integer-coded design variable crossover and mutation, standard genetic
algorithm operations, were developed.  Crossover of an integer design
variable across two population members with integer values $z_1$ and
$z_2$ is simply a random integer in the interval, $[z_1, z_2]$.
Similarly, mutation of an integer design variable is a random integer
number selected between the lower and upper bounds.

The user selected parameters and initial conditions for the NSGA-II
algorithm were selected to foster a broad and fluid search of the entire
tradespace.  A population size of 30 was initialized with Latin
Hypercube sampling across the range of permissible design variable
values from the lower bound to the upper bound.  Thus, this was truly a
\textit{tabula rasa} (blank slate) approach to substructure design.  The
probability of crossover during the optimization was set to 0.9 and the
probably of mutation was 0.4.  Both of these values are considered high,
in the context of genetic algorithm scientific literature, but again
enabled a broad and fluid search of the tradespace.

By design, the NSGA-II algorithm is well-suited to global optimization
by traversing a broad span and combination of design variable values.
However, for single-objective optimization (such as mass or cost
minimization), the crossover and mutation operations, with their
inherent use of random numbers, are inefficient at searching a design
space neighborhood for local minima.  Thus, after the genetic algorithm
terminated, another local optimization was performed.

Local optimization was performed with the Nelder-Mead simplex algorithm
\citep{neldermead}.  The Nelder–Mead method, also referred to as the
downhill simplex method, is a derivative-free heuristic optimization
method.  It uses a simplex, a shape with $n+1$ vertices in
$n$-dimensional space (e.g. a triangle on a plane or a tetrahedron is
three-dimensional space), to approximate the objective function
behavior.  The vertices of the simplex, the test points, are then
reflected, expanded, or contracted to move towards the optimal point.
The process terminates when the simplex becomes sufficiently small or
the test points have nearly identical performance.  As with the genetic
algorithm, an in-house implementation of the Nelder-Mead algorithm was used
to enable multi-threading and constraint handling via the chosen penalty
method.  To focus the algorithm purely on local optimization, and not
global exploration of other possible floating substructure
configurations, only continuous design variables were used.  Integer
design variables such as the number of offset columns or mooring lines
were frozen at their values chosen by the NSGA-II algorithm.
Additionally, since this approach is best suited for finding local
minima near the initial condition, it is the only algorithm used for the
design sensitivity portions of the analysis.

User parameters values and simplex initialization methodology was
borrowed from the open source SciPy implementation of the Nelder-Mead
algorithm.  Initialization was done where each vertex, or test point, in
the simplex perturbed one of the design variables by 5\%, with the
remaining design variables left at their baseline value.  As a larger
design problem with over 100 design variables (for the semisubmersible),
the parameters that govern the iterative modification of the simplex
were assigned dimension-dependent values,
\begin{equation}
\alpha = 1.0, \quad \beta = 0.75 - \frac{1}{2n}, \quad \gamma = 1+\frac{2}{n} 
\quad \delta = 1 - \frac{1}{n};
\end{equation}
where the parameters are $\alpha$ for reflection, $\beta$ for
contraction, $\gamma$ for expansion and $\delta$ for shrinkage. These
values differ from those originally used by \citet{neldermead}, but are
recommended for high-dimensional optimization problems.

\subsection{Rationale for Derivative-Free Algorithms}
Derivative-free optimization algorithms were chosen for a few reasons,
despite their known performance drawbacks in terms of wall-clock time.
First, to do a complete configuration optimization of the substructure,
a mixed-integer capable algorithm is required.  No gradient-based
optimization algorithm is capable of handling these types of variables
directly (unless a rounding approximation is used).  This was the
primary reason for the selection of a genetic algorithm for the global
design space search and optimization step.

Another reason for the selection of derivative-free algorithms is that
the analysis flow uses a number of third-party, black box tools or
algorithms that do not come with analytical gradients.  This includes
Frame3DD, MAP++, and some of the API 2U procedures that rely on roots of
nonlinear equations.  Thus, gradient-based optimization algorithms would
be forced to use finite difference approximations around these tools at
the very least.  However, derivatives approximated with finite
differences are expensive to compute accurately.  If computed
inaccurately, for the sake of reducing computational time, finite
difference derivatives can easily lead an optimization algorithm astray,
especially in highly nonlinear or tightly constrained regions of the
design space.  This is another reason for the use of
derivative-free algorithms, even when conducting local neighborhood
design space optimization and/or sensitivity studies.

\subsection{Constraint Handling}
The standard optimization problem statement is modified in this
application to use a penalty approach for constraint handling.  Instead
of treating the set of constraints, $\mbf{g}(\mbf{x})$, as
\textit{hard}-constraints that absolutely must be satisfied, a penalty
method treats them as \textit{soft}-constraints, that are considered in
conjunction with the objective function.  This is a common approach that
involves summation of a constraint violations into a single metric.
This metric is then scaled and added or multiplied to the objective
function value of each solution.  \citet{yeniay05} provides a nice
summary of the use of penalty methods in genetic algorithms.  In our
implementation, an \textit{adaptive} penalty approach is used where the
scaling of the constraint violation summation is dependent on the value
of the objective function of the best solution in the population at
every iteration.  With this approach, the problem formulation becomes,
\begin{equation}
\begin{array}{ll}
  \min & \Phi(\mbf{x}) = f\left(\mbf{x}\right) + p\left(\mbf{x}\right)\\
  \text{subject to}& \mbf{x} \in \mbf{X}\\
  \end{array}
\end{equation}
where $\Phi(\mbf{x})$ is the new objective function and $p(\mbf{x})$ is
the penalty function.  This function is configured as a summation of
constraint violations only (constraints that are satisfied add zero to the summation),
\begin{equation} \label{eqn:penalty}
p\left(\mbf{x}\right) = \lambda \sum_i^m \max\left[0,g_i
  \left(\mbf{x}\right)\right];
\qquad \lambda = 10^{\text{floor} \left( \log_{10} \min\left[f\left(x_1\right)
  \ldots f\left(x_k\right) \right] \right)}
\end{equation}
and $\lambda$ is the adaptive scaling parameter, which is essentially
set to the next order of magnitude above that of the objective function
for the best performing solution in every iteration/generation.

A penalty approach is used in our application of conceptual design of a
floating offshore wind energy system because its advantages outweigh its
challenges.  A standard drawback of the penalty approach is that it can
be difficult to find a problem independent method for constraint
summation and scaling parameters.  First among the advantages of a
penalty approach is the fluid searching of the tradespace between the
pockets of feasibility.  Meaning, our tradespace is like swiss cheese,
with pockets of feasibility for spars, semisubmersibles, TLPs, and their
hybrids.  In between these pockets are many infeasible designs that
violate the many constraints involved.  A penalty approach allows the
optimizer to cross fluidly from one pocket of feasibility to another.
The second advantage is that a penalty approach can identify promising
designs even when they are not fully compliant with all constraints.
Essentially, the constraints involved are truly \textit{soft}
constraints and not \textit{hard} constraints.  In practice, small
violations of constraints in the conceptual design can typically be
mended during the detailed design phase.  Furthermore, the low-fidelity
representation of the physics likely fails to capture all of the nuances
involved in the constraint evaluation, so promising designs should not
be discarded until fully vetted by higher-fidelity models.

It should be noted that the penalty approach described weights all
constraint violations equally.  Meaning, all constraint violations are
summed together into $p(x)$ when in truth some violations are softer or
harder than others.  A modified approach would be to have a different
scaling for each constraint function, so $\lambda$ would become
$\lambda_i$ and move inside the summation in Equation \ref{eqn:penalty}.
This has been considered for this work, but not yet enforced.

\subsection{Design Variables}
In WISDEM, via OpenMDAO, any input parameter can be designated a design
variable.  The design variables used in this study focused on the
geometric specification of the floating substructure and mooring
subsystem.  Slightly different design variables and bounds were used for
spar, semisubmersible, and TLP optimizations.  The complete listing of
the design variables for each optimization configuration is shown in
Table \ref{tbl:designvar}.  Note that the integer design variables were
only used in the global optimization with the genetic algorithm, not the
local search with the simplex algorithm.

\begin{table}[htbp] \begin{center}
    \caption{Standard design variables, their size, and units used for
      optimization in \textit{FloatingSE}.  Note that $n_s$ denotes the
      number of sections in the column discretization.}
    \label{tbl:designvar}
{\footnotesize
  \begin{tabular}{ l c l c l } \hline
    \textbf{Variable} & \textbf{Units} & \textbf{Type} & \textbf{Bounds} & \textbf{Comments} \\ \hline \hline
    Main col section height & \unit{$m$} & Float array ($n_s$) & 0.1--50 &\\
    Main col outer diameter & \unit{$m$} & Float array ($n_s+1$) &2.1--40 &\\
    Main col wall thickness & \unit{$m$} & Float array ($n_s+1$)  &0.001--0.5 &\\
    Main col freeboard & \unit{$m$} & Float scalar &0--50 &\\
    Main col stiffener web height & \unit{$m$} & Float array ($n_s$) &0.01--1 &\\
    Main col stiffener web thickness & \unit{$m$} & Float array ($n_s$) &0.001--0.5 &\\
    Main col stiffener flange width & \unit{$m$} & Float array ($n_s$) &0.01--5 &\\
    Main col stiffener flange thickness & \unit{$m$} & Float array ($n_s$) &0.001--0.5 &\\
    Main col stiffener spacing & \unit{$m$} & Float array ($n_s$) &0.1--100 &\\
    Main col permanent ballast height & \unit{$m$} & Float scalar &0.1--50 &\\
    Main col buoyancy tank diameter & \unit{$m$} & Float scalar &0--50 &\\
    Main col buoyancy tank height & \unit{$m$} & Float scalar &0--20 &\\
    Main col buoyancy tank location (fraction) && Float scalar &0--1 &\\
    \hline
    Number of offset cols && Integer scalar & 3-5 & semi only\\
    Offset col section height & \unit{$m$} & Float array ($n_s$) &0.1--50&semi only\\
    Offset col outer diameter & \unit{$m$} & Float array ($n_s+1$)&1.1--40&semi only\\
    Offset col wall thickness & \unit{$m$} & Float array ($n_s+1$) &0.001--0.5&semi only\\
    Offset col freeboard & \unit{$m$} & Float scalar &2--15 &semi only\\
    Offset col stiffener web height & \unit{$m$} & Float array ($n_s$) &0.01--1&semi only\\
    Offset col stiffener web thickness & \unit{$m$} & Float array ($n_s$) & 0.001--0.5&semi only\\
    Offset col stiffener flange width & \unit{$m$} & Float array ($n_s$) & 0.01--5&semi only\\
    Offset col stiffener flange thickness & \unit{$m$} & Float array ($n_s$) & 0.001--0.5&semi only\\
    Offset col stiffener spacing & \unit{$m$} & Float array ($n_s$) &0.01--100&semi only\\
    Offset col permanent ballast height & \unit{$m$} & Float scalar &0.1--50&semi only\\
    Offset col buoyancy tank diameter & \unit{$m$} & Float scalar &0--50 & semi only\\
    Offset col buoyancy tank height & \unit{$m$} & Float scalar &0--20 & semi only\\
    Offset col buoyancy tank location (fraction) && Float scalar & 0--1 & semi only\\
    Radius to offset col & \unit{$m$} & Float scalar &5--100 &semi only\\
    \hline
    Pontoon outer diameter & \unit{$m$} & Float scalar &0.1--10 &semi only\\
    Pontoon wall thickness & \unit{$m$} & Float scalar &0.01--1 &semi only\\
    Lower main-offset pontoons && Integer scalar & 0--1 & semi only\\
    Upper main-offset pontoons && Integer scalar & 0--1 & semi only\\
    Cross main-offset pontoons && Integer scalar & 0--1 & semi only\\
    Lower offset ring pontoons && Integer scalar & 0--1 & semi only\\
    Upper offset ring pontoons && Integer scalar & 0--1 & semi only\\
    Outer V-pontoons && Integer scalar & 0--1 & semi only\\
    Main col pontoon attach lower (fraction) && Float scalar & 0--0.5&semi only\\
    Main col pontoon attach upper (fraction) && Float scalar & 0.5--1&semi only\\
    \hline
    Fairlead (fraction) && Float scalar &0--1 &\\
    Fairlead offset from col & \unit{$m$} & Float scalar & 5--30 & TLP only\\
    Fairlead pontoon diameter & \unit{$m$} & Float scalar & 0.1--10 &\\
    Fairlead pontoon wall thickness & \unit{$m$} & Float scalar &0.001--1 &\\
    Number of mooring connections && Integer scalar &3--5 &\\
    Mooring lines per connection && Integer scalar &1--3 &\\
    Mooring diameter & \unit{$m$} & Float scalar &0.05--2 &\\
    Mooring line length & \unit{$m$} & Float scalar &0--3000 &TLP 10--300\\
    Anchor distance & \unit{$m$} & Float scalar &0--5000 & TLP 20-100\\
  \hline \end{tabular}
}
\end{center} \end{table}


\subsection{Constraints}
Due to the many design variables, permutations of settings, and applied
physics, there are many constraints that must be applied for an
optimization to close.  The constraints capture both physical
limitations, such as column buckling, but also inject industry
standards, guidelines, and lessons learned from engineering experience
into the optimization.  As described in Section \ref{sec:intro}, this is
a critically important element in building a MDAO framework for
conceptual design that yields feasible results worth interrogating
further with higher-fidelity tools.  The constraints used in the
substructure design optimization and sensitivity studies are listed in
Table \ref{tbl:constraints}.  Where appropriate, some of the constraint
values differ from one type of substructure to another.

\begin{table}[htbp] \begin{center}
    \caption{Optimization constraints used in \textit{FloatingSE}.}
    \label{tbl:constraints}
    {\footnotesize
  \begin{tabular}{ c l c l} \hline
    \textbf{Lower} & \textbf{Name} & \textbf{Upper} & \textbf{Comments}\\
\hline \hline
 & \textbf{Tower / Main / Offset Columns} &  & \\
 & Eurocode global buckling & 1.0 & \\
 & Eurocode shell buckling & 1.0 & \\
 & Eurocode stress limit & 1.0 & \\
  & Manufacturability & 0.5 & Taper ratio limit\\
  120.0 & Weld-ability &  & Diameter:thickness ratio limit\\
\hline & \textbf{Main / Offset Columns} &  & \\
 & Draft ratio & 1.0 & Ratio of draft to max value (spar \unit[200]{m}, semi/TLP \unit[30]{m})\\
 & API 2U general buckling- axial loads & 1.0 & \\
 & API 2U local buckling- axial loads & 1.0 & \\
 & API 2U general buckling- external loads & 1.0 & \\
 & API 2U local buckling- external loads & 1.0 & \\
 & Wave height:freeboard ratio & 1.0 & Maximum wave height relative to freeboard\\
  1.0 & Stiffener flange compactness &  & \\
  1.0 & Stiffener web compactness &  & \\
 & Stiffener flange spacing ratio & 1.0 & Stiffener spacing relative to flange width\\
 & Stiffener radius ratio & 0.50 & Stiffener height relative to diameter\\
\hline & \textbf{Offset Columns} &  & \textit{Semi only}\\
  0.0 & Heel freeboard margin &  & Height required to stay above waterline at max heel\\
  0.0 & Heel draft margin &  & Draft required to stay submerged at max heel\\
\hline & \textbf{Pontoons} &  & \textit{Semi only}\\
 & Eurocode stress limit & 1.0 &\\
\hline & \textbf{Tower} &  & \\
  -0.01 & Hub height error & 0.01 &\\
\hline & \textbf{Mooring} &  & \\
  0.0 & Axial stress limit & 1.0 &\\
 & Line length limit & 1.0 & Loss of tension or catenary hang\\
 & Heel moment ratio & 1.0 & Ratio of overturning moment to restoring moment\\
 & Surge force ratio & 1.0 & Ratio of surge force to restoring force\\
\hline & \textbf{Geometry} &  & \\
  1.0 & Main-offset spacing &  & Minimum spacing between main and offset columns \\
  0.0 & Fairlead:draft ratio & 1.0 &\\
  0.0 & Nacelle transition buffer &  & Tower diameter limit at nacelle junction\\
  -1.0 & Tower transition buffer & 1.0 & Diameter consistency at freeboard point\\
\hline & \textbf{Stability} &  & \\
  0.10 & Metacentric height &  & \textit{Not applied to TLPs}\\
  1.0 & Wave-Eigenmode boundary (upper) &  & Natural frequencies below wave frequency range\\
 & Wave-Eigenmode boundary (lower) & 1.0 & Natural frequencies above wave frequency range\\
  0.0 & Water ballast height limit & 1.0 & \\
  0.0 & Water ballast mass &  & Neutral buoyancy\\
    \hline \end{tabular}
  }
\end{center} \end{table}
