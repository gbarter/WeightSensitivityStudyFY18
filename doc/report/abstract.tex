\chapter*{Acknowledgements}
The original version of this model was developed by Senu Sirnivas, who
deserves credit for laying out the methodology described here.  The
authors would also like to thank Patrick Gilman, Alana Duerr, Gary
Norton, Daniel Beals, Eric Lantz, and Jason Jonkman for their valuable
input in reviewing results and drafts of this paper.

This work was supported by the U.S. Department of Energy (DOE) under
Contract Number DE-AC36-08GO28308 with NREL. Funding for the work was
provided by the DOE Office of Energy Efficiency and Renewable Energy,
Wind Energy Technologies Office.

\chapter*{Abstract}
This work introduces \textit{FloatingSE}, the floating offshore wind
turbine substructure cost, sizing, and analysis module in the Wind-Plant
Integrated System Design \& Engineering Model (WISDEM) framework.  The
tool generalizes its geometry parameterization so as to use the same set
of variables to describe spars, semisubmersibles, tension leg platforms
(TLPs), and hybrids of those archetypes.  Design evaluation is done by
the application of existing codes and standards, hydrostatic principals,
and simple beam finite element static structural analysis.  With other
multidisciplinary WISDEM modules, the inclusion of \textit{FloatingSE}
enables the vertical simulation of a floating offshore wind turbine,
even a whole plant if desired.  To showcase this capability, a two-step
optimized-based analysis was carried out.  First, three different
substructures, a spar, semisubmersible, and TLP, were designed beneath a
\unit[10]{MW} reference turbine. Second, a design sensitivity study was
conducted where mass in the nacelle was parametrically removed, to
simulate the addition of a novel drivetrain or generator technology, and
the design re-optimized.  The derived sensitivities were used to
ascertain the break-even cost rate of the new technology that reduces
the drivetrain mass, but at a cost premium (\$1,000 for the spar, \$450
for the semisubmersible, and \$100 for the TLP).  Another noteworthy
observation is that mass is poor surrogate for cost in optimization
studies, a convention frequently used in conceptual design studies.  Due
to the many simplifying assumptions and low-fidelity analysis, the
optimized design and sensitivity values come with many caveats and are
subject to change following future development.
