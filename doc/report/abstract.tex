\chapter*{Acknowledgements}
The original version of this model was developed by Senu Sirnivas, who
deserves credit for laying out the methodology described here.  The
authors would also like to thank Patrick Gilman, Alana Duerr, Gary
Norton, Daniel Beals, Eric Lantz, and Jason Jonkman for their valuable
input in reviewing results and drafts of this paper.

This work was supported by the U.S. Department of Energy (DOE) under
Contract Number DE-AC36-08GO28308 with NREL. Funding for the work was
provided by the DOE Office of Energy Efficiency and Renewable Energy,
Wind Energy Technologies Office.

\chapter*{Abstract}
This work introduces \textit{FloatingSE}, the floating offshore wind
turbine substructure cost, sizing, and analysis module in the Wind-Plant
Integrated System Design \& Engineering Model (WISDEM) framework.  The
tool generalizes its geometry parameterization enabling the same set of
variables to describe spars, semisubmersibles, tension leg platforms
(TLPs), and hybrids of those archetypes.  Design evaluation is done by
the application of existing codes and standards, hydrostatic principals,
and simple beam finite element static structural analysis.  With other
multidisciplinary WISDEM modules, the inclusion of \textit{FloatingSE}
enables the holistic simulation of a floating offshore wind turbine.  To
showcase this capability, a two-step analysis is presented.  First,
three different substructures, a spar, semisubmersible, and TLP, were
designed via optimization with design variables and constraints to
support a \unit[10]{MW} reference turbine. Second, a design sensitivity
study was conducted where mass in the nacelle was parametrically
removed, to simulate the weight reduction that may be achieved from some
novel drivetrain or generator technologies, and the design re-optimized
to ascertain the corresponding reduction in mass and cost in the
substructure.  Based on these sensitivities for drivetrain mass, we
computed an incremental cost premium for the more expensive, but
lighter-weight technology.  Up to \$1,000/kg for the spar, \$450/kg for
the semisubmersible, and \$100/kg for the TLP could be incurred and
result in a levelized cost that is equal to or less than the reference
design.  Another noteworthy observation is that mass is a poor surrogate
for cost in optimization studies, if applied blindly, a convention
frequently used in conceptual design studies.  Due to the many
simplifying assumptions and low-fidelity analysis, the optimized design
and sensitivity values come with many caveats and are subject to change
following future development.  Planned improvements in the
\textit{FloatingSE} include the incorporation of a simple hydrodynamic
physics and evaluation by additional Design Load Cases (DLCs).
